% Created 2017-03-21 二 20:24
\documentclass[11pt]{article}
\usepackage[utf8]{inputenc}
\usepackage[T1]{fontenc}
\usepackage{fixltx2e}
\usepackage{graphicx}
\usepackage{longtable}
\usepackage{float}
\usepackage{wrapfig}
\usepackage{rotating}
\usepackage[normalem]{ulem}
\usepackage{amsmath}
\usepackage{textcomp}
\usepackage{marvosym}
\usepackage{wasysym}
\usepackage{amssymb}
\usepackage{hyperref}
\tolerance=1000
\author{Qiu Wei}
\date{\today}
\title{README}
\hypersetup{
  pdfkeywords={},
  pdfsubject={},
  pdfcreator={Emacs 24.5.1 (Org mode 8.2.10)}}
\begin{document}

\maketitle
\tableofcontents

\section{Arxiv-Spider}
\label{sec-1}
This is a spider (or scrawl) to download papers in computer science area in \href{http://arxiv.org/}{arxiv.org}.
The paper will be used in our prp project.
\section{How to Use It}
\label{sec-2}
First,ensure you have installed python3 in your computer,then download the source code
from github:
\begin{verbatim}
$ git clone https://github.com/cjqw/arxiv-spider ~/arxiv-spider
\end{verbatim}
Change the first line to specify your python3 path.
Run the program directly and it will download 10 papers by default.
\begin{verbatim}
$ chmod a+x spider.py
$ ./spider.py
\end{verbatim}
To doanload more papers, change the value of max\_pdf to whatever you want.
% Emacs 24.5.1 (Org mode 8.2.10)
\end{document}